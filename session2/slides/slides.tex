\documentclass{beamer}

\let\phi\varphi
\newcommand{\field}{\mathbf{F}}
\newcommand{\impliesn}[1]{\underset{#1}{\implies}}
\newcommand{\iffn}[1]{\underset{#1}{\iff}}
\newcommand{\impliesp}{\impliesn{p}}
\newcommand{\iffp}{\iffn{p}}
\newcommand{\impliespp}{\impliesn{p'}}
\newcommand{\iffpp}{\iffn{p'}}
\newcommand{\ex}{\mathbf{E}}


\mode<presentation>
{
\usetheme{default}
}
\setbeamertemplate{navigation symbols}{}
\usecolortheme[rgb={0.13,0.28,0.59}]{structure}
\setbeamertemplate{itemize subitem}{--}
\setbeamertemplate{frametitle} {
        \begin{center}
          {\large\bf \insertframetitle}
        \end{center}
}

\newcommand\footlineon{
  \setbeamertemplate{footline} {
    \begin{beamercolorbox}[ht=2.5ex,dp=1.125ex,leftskip=.8cm,rightskip=.6cm]{structure}
      \footnotesize \insertsection
      \hfill
      {\insertframenumber}
    \end{beamercolorbox}
    \vskip 0.45cm
  }
}
\footlineon

\AtBeginSection[]
% {       
%         \begin{frame}<beamer> 
                \frametitle{Outline} 
                \tableofcontents[currentsection,currentsubsection]
        \end{frame}
}

% \usepackage{subcaption}
\usepackage{amsmath,amssymb,amsthm}
\usepackage{xcolor}
\usepackage{graphicx}
\usepackage{tikz}
\usetikzlibrary{matrix, positioning}

\makeatletter
\def\mathcolor#1#{\@mathcolor{#1}}
\def\@mathcolor#1#2#3{%
  \protect\leavevmode
  \begingroup
    \color#1{#2}#3%
  \endgroup
}
\makeatother

\title{Common Tools in Succinct Proofs}
\author{Guillermo Angeris \and Alex Evans}
\date{Session 2, SPLA Study Club}

\newcommand{\ones}{\mathbf 1}
\newcommand{\reals}{{\mbox{\bf R}}}
\newcommand{\integers}{{\mbox{\bf Z}}}
\newcommand{\symm}{{\mbox{\bf S}}}  % symmetric matrices

\newcommand{\nullspace}{{\mathcal N}}
\newcommand{\range}{{\mathcal R}}
\newcommand{\Rank}{\mathop{\bf Rank}}
\newcommand{\tr}{\mathop{\bf tr}}
\newcommand{\diag}{\mathop{\bf diag}}
\newcommand{\card}{\mathop{\bf card}}
\newcommand{\rank}{\mathop{\bf rank}}
\newcommand{\conv}{\mathop{\bf conv}}
\newcommand{\prox}{\mathbf{prox}}

\newcommand{\Expect}{\mathop{\bf E{}}}
\newcommand{\Prob}{\mathop{\bf Prob}}
\newcommand{\Co}{{\mathop {\bf Co}}} % convex hull
\newcommand{\dist}{\mathop{\bf dist{}}}
%\newcommand{\argmin}{\mathop{\rm argmin}}
%\newcommand{\argmax}{\mathop{\rm argmax}}
\DeclareMathOperator*{\argmin}{argmin}
\DeclareMathOperator*{\argmax}{argmax}
\newcommand{\epi}{\mathop{\bf epi}} % epigraph
\newcommand{\Vol}{\mathop{\bf vol}}
\newcommand{\dom}{\mathop{\bf dom}} % domain
\newcommand{\intr}{\mathop{\bf int}}
\newcommand{\sign}{\mathop{\bf sign}}

\newcommand{\cf}{{\it cf.}}
\newcommand{\eg}{{\it e.g.}}
\newcommand{\ie}{{\it i.e.}}
\newcommand{\etc}{{\it etc.}}

\newcommand{\BEAS}{\begin{eqnarray*}}
\newcommand{\EEAS}{\end{eqnarray*}}
\newcommand{\BEA}{\begin{eqnarray}}
\newcommand{\EEA}{\end{eqnarray}}
\newcommand{\BEQ}{\begin{equation}}
\newcommand{\EEQ}{\end{equation}}
\newcommand{\BIT}{\begin{itemize}}
\newcommand{\EIT}{\end{itemize}}

\begin{document}
	\begin{frame}
		\titlepage
	\end{frame}
	
    \section{Overview}
    \begin{frame}
        \frametitle{Quick recap}
        \begin{itemize}\itemsep=12pt
            \item Last session was mostly pure math
            \item This one, we will start doing real(-ish!) stuff
            \pause
            \item Let's put some of the previous tools to the test!
        \end{itemize}
	\end{frame}

    \begin{frame}
        \frametitle{High level overview}
        \begin{itemize}\itemsep=12pt
            \item We're going to start by discussing \emph{models}
            \item Then progress onto specific tools
            \item Will mostly deal with `exact' checks
            \item (Leave `sparse' checks for homework/paper)
        \end{itemize}
    \end{frame}

    \begin{frame}
        \frametitle{In other words...}
        \pause
        \begin{center}
            {\Large Let's build up the toolbox!} 
        \end{center}
    \end{frame}

    \section{Randomness}
    \begin{frame}
        \frametitle{Why does randomness help?}
        \begin{itemize}\itemsep=12pt
            \item Want to certify that some given vector $x \in \field^n$ is \emph{sparse}
            \item Say check that $\le 10\%$ of entries are nonzero
            \pause
            \item Can start checking until we see $90\%$ of entries are nonzero
            \item If the vector is large, this is not great
        \end{itemize}
    \end{frame}

    \begin{frame}
        \frametitle{Randomness (continued)}
        \begin{itemize}\itemsep=12pt
            \item Checking every element requires $\sim n$ checks
            \item If $n = 2^{20}$, that's a lot of queries
            \pause
            \item If instead we care that $\le 10\%$ of entries are nonzero
            \item Can be certain (up to $2^{-100}$ probability) with $< 700$ queries!
            \pause
            \item See homework :)
        \end{itemize}
    \end{frame}

    \section{Models}
    \begin{frame}
        \frametitle{What are models?}
        \begin{itemize}\itemsep=12pt
            \item \emph{Models} are how we `encapsulate' interactions
            \item Along with cryptographic tools needed
            \pause
            \item We will discuss two models of interaction
        \end{itemize}
    \end{frame}

    \begin{frame}
        \frametitle{Direct access model}
        \begin{itemize}\itemsep=12pt
            \item The first is the \emph{direct model}
            \pause
            \item There exists a vector $x \in \field^n$
            \item We would like to verify some claim $Q$ about $x$
            \item Can query entries of $x$ (\eg, can query $x_i$)
        \end{itemize}
    \end{frame}


    \begin{frame}
        \frametitle{Coding model}
        \begin{itemize}\itemsep=12pt
            \item The second is the \emph{coding model}
            \pause
            \item We have a (known) code matrix $G\in \field^{m \times n}$
            \item There exists some message $x \in \field^n$
            \item Would like to verify some claim $Q$ about $x$
            \item Can query individual \emph{symbols} of the encoded message $Gx$
        \end{itemize}
    \end{frame}

    \begin{frame}
        \frametitle{Truthfulness and commitments}
        \begin{itemize}\itemsep=12pt
            \item By assumption, the models require truthfulness
            \item In general, this is achieved via cryptographic techniques
            \item Ex.: the direct model can be achieved via a Merkle tree
            \item We elide this here! (But it is fascinating and worth studying)
        \end{itemize}
    \end{frame}

    \section{Exact checks}
    \begin{frame}
        \frametitle{Exact check model}
        \begin{itemize}\itemsep=12pt
            \item Exact checks work in the coding model
            \item Very hard in the direct access model
            \pause
            \item See homework exercises!
        \end{itemize}
    \end{frame}

    \begin{frame}
        \frametitle{Set up}
        \begin{itemize}\itemsep=12pt
            \item For the remainder of section: fix a code matrix $G \in \field^{m \times n}$
            \item Code has distance $d > 0$
            \item And $r$ will be uniformly drawn from $\{1,\dots, m\}$
            \item (Think of $r$ as drawing a uniformly random row of $G$)
        \end{itemize}
    \end{frame}

    \begin{frame}
        \frametitle{Zero check}
        \begin{itemize}\itemsep=12pt
            \item Let's start with the simplest check
            \item Given some $x \in \field^n$, would like to check $x = 0$
            \pause
            \item Randomly sample $r$ and then check if $(Gx)_r = 0$
            \pause
            \item We can write this as
            \[
                (Gx)_r = 0 \quad \impliesp \quad x = 0
            \]
            \item Here, $p \le 1 - d/m$
        \end{itemize}
    \end{frame}

    \begin{frame}
        \frametitle{Intuition}
        \begin{figure}
            \centering
            \includegraphics[width=.7\textheight]{images/zerocheck.pdf}
        \end{figure}
    \end{frame}

    \begin{frame}
        \frametitle{Discussion}
        \begin{itemize}\itemsep=12pt
            \item Equivalent to 1D Schwartz--Zippel lemma
            \item But for general linear codes!
            \item (Schwartz--Zippel is the special case of Reed--Solomon)
        \end{itemize}
    \end{frame}

    \begin{frame}
        \frametitle{Matrix zero check}
        \begin{itemize}\itemsep=12pt
            \item Instead of a vector $x$, let's check a matrix $X \in \field^{k \times n}$
            \item Ask the same question: does $X = 0$?
            \pause
            \item Idea: encode each row of $X$ and check if \emph{that} is zero!
            \pause
            \item Let $\tilde x_i$ denote $i$th row, then
            \[
                (G\tilde x_1)_r = 0, ~ \dots, ~ (G\tilde x_n)_r = 0\quad  \impliesp \quad X = 0
            \]
            \vspace{-2em}
            \pause
            \item What should $p$ be here? (See homework 1, exercise 5!)
        \end{itemize}
    \end{frame}


    \begin{frame}
        \frametitle{Matrix zero check (equiv.)}
        \begin{itemize}\itemsep=12pt
            \item Another (identical) way of writing this:
            \begin{enumerate}\itemsep=8pt
                \item Pick random row $r$ from $G$
                \item Use row to take linear combination of cols $x_1, \dots, x_n$ of $X$
                \item Check if this linear combination is zero
            \end{enumerate}
            \pause
            \item We can write this as:
            \[
                G_{r1}x_1 + \dots + G_{rn} x_n = 0 \quad \impliesp \quad X = 0
            \]
        \end{itemize}
    \end{frame}

    \begin{frame}
        \frametitle{Reduced matrix zero check}
        \begin{itemize}\itemsep=12pt
            \item Let's get funky!
            \pause
            \item The vector zero check goes
            \[
                \text{Vector}\stackrel{?}{=}0 \quad \to \quad \text{Scalar}\stackrel{?}{=}0
            \]
            \vspace{-2em}
            \pause
            \item The matrix zero check goes
            \[
                \text{Matrix}\stackrel{?}{=}0 \quad \to \quad \text{Vector}\stackrel{?}{=}0
            \]
            \vspace{-2em}
            \pause
            \item Can we `put them together'? \pause (Yes! See homework)
        \end{itemize}
    \end{frame}

    \begin{frame}
        \frametitle{Aside: reduced matrix zero check}
        \begin{itemize}\itemsep=12pt
            \item This check is a Schwartz--Zippel generalization
            \item Indeed, we get the same bounds for Reed--Solomon codes
            \item (But can get better ones! See~\S3.1.3 of the paper)
            \item Shows Schwartz--Zippel is not tight (by a very small factor...)
        \end{itemize}
    \end{frame}

    \begin{frame}
        \frametitle{Vector subspace check}
        \begin{itemize}\itemsep=12pt
            \item The `final' boss
            \pause
            \item Let's start with some matrix $X \in \field^{k \times n}$
            \item Want to check if all columns of $X$ are in a subspace $V \subseteq \field^k$
            \item Can we do this cheaply?
        \end{itemize}
    \end{frame}

    \begin{frame}
        \frametitle{Vector subspace check}
        \begin{itemize}\itemsep=12pt
            \item Let's do something similar to matrix zero check:
            \begin{enumerate}\itemsep=8pt
                \item Pick random row $r$
                \item Use row for linear combination of columns of $X$
                \item Check if linear combination lies in $V$
            \end{enumerate}
            \pause
            \item We can write this as
            \[
                G_{r1}x_1 + \dots + G_{rn} x_n \in V \quad\impliesp\quad x_i \in V,~~i=1,\dots,n
            \]
        \end{itemize}
    \end{frame}

    \begin{frame}
        \frametitle{Proof via probabilistic implications}
        \begin{itemize}\itemsep=12pt
            \item Let $C$ be parity check matrix for $V$
            \item Then
            \[
                G_{r1}x_1 + \dots + G_{rn} x_n \in V ~~\text{implies}~~ C(G_{r1}x_1 + \dots + G_{rn} x_n) = 0
            \]
            \vspace{-2em}
            \pause
            \item But, by linearity (see homework 1, problem 1)
            \[
                0 = C(G_{r1}x_1 + \dots + G_{rn} x_n) = G_{r1}(Cx_1) + \dots + G_{rn}(Cx_n)
            \]
            \item Note the last quantity!
        \end{itemize}
    \end{frame}

    \begin{frame}
        \frametitle{Proof via probabilistic implications (cont.)}
        \begin{itemize}\itemsep=12pt
            \item From before,
            \[
                G_{r1}(Cx_1) + \dots + G_{rn}(Cx_n) = 0
            \]
            \item But, this is just the matrix zero check! \pause So,
            \[
                G_{r1}(Cx_1) + \dots + G_{rn}(Cx_n) = 0 \quad\impliesp\quad Cx_1 =\dots= Cx_n = 0
            \]
            \vspace{-2em}
            \pause
            \item By definition of parity check matrix then
            \[
                Cx_1 = \dots = Cx_n = 0 \quad \text{implies} \quad x_1, \dots, x_n \in V
            \]
        \end{itemize}
    \end{frame}

    \begin{frame}
        \frametitle{Putting it all together}
        \begin{itemize}\itemsep=12pt
            \item We only used one implication with error $p$ (matrix zero check)
            \item Rest were all `standard' implications (\ie, zero error)
            \item This means that we can write:
            \[
                G_{r1}x_1 + \dots + G_{rn} x_n \in V \quad\impliesp\quad x_i \in V,~~i=1,\dots,n
            \]
            \item Where $p$ is the same as the matrix zero check from before
        \end{itemize}
    \end{frame}
    \section{Sparse checks}
    \begin{frame}
        \frametitle{Analogues of exact checks}
        \begin{itemize}\itemsep=12pt
            \item There is a sparse analogue of each exact check
            \item We will not explore proofs here (as they are more complicated)
            \item But check the paper (and the homework!) for some of these
        \end{itemize}
    \end{frame}

    \begin{frame}
        \frametitle{Sparse checks}
        \begin{itemize}\itemsep=12pt
            \item We will use two types of sparse checks for next lecture
            \item One will be in homework (standard sparsity check)
            \item The other will have a proof outline next lecture
            \pause
            \item It's good to be comfortable with notation before that :)
        \end{itemize}
    \end{frame}

%    \begin{frame}
%        \frametitle{Proof outline ($n=2$)}
%        \begin{itemize}\itemsep=12pt
%            \item Split $d(X, V) \ge d'/2$ up into two cases:
%            \vspace{.3em}
%            \begin{enumerate}\itemsep=8pt
%                \item There exists $v \in \field^2$ with $\|Xv - V\|_0 > 2q$
%                \item For every $v \in \field^2$: $\|Xv - V\|_0 \le 2q$
%            \end{enumerate}
%            \pause
%            \item Case (1) we `catch' with high probability (`easy' case)
%            \item Case (2) means $2q < d'/2$; can find $y \in V$ closest to $Xv$
%            \item Second case reduces to matrix sparsity check!
%            \pause
%            \item See paper for details!
%        \end{itemize}
%    \end{frame}

    \section{Where to?}
    \begin{frame}
        \frametitle{Summary}
        \begin{itemize}\itemsep=12pt
            \item Rewrote a bunch of tools used in many papers!
            \item Some maybe felt familiar
            \item But did all of them in very short order
        \end{itemize}
    \end{frame}

    \begin{frame}
        \frametitle{Next lecture}
        \begin{itemize}\itemsep=12pt
            \item We will see how to apply these tools to explain FRI
            \item (As a quasi-`capstone' project)
            \item But many more applications exist :)
        \end{itemize}
    \end{frame}
\end{document}
